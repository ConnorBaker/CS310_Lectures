% Declare type of document
\documentclass[10pt]{article}

% Import Packages
\usepackage[utf8]{inputenc}

% Commonly used math symbols and fonts
\usepackage[mathscr]{euscript}
\usepackage{amsfonts,amsmath,amssymb,amsthm}
\usepackage{mathtools,mathdots}

% Better looking default font
\usepackage{lmodern}

% itemize environment
\usepackage{enumitem}

% Array, longtable, and booktabs
\usepackage{array}
\usepackage{longtable}
\usepackage{booktabs}

% Caption package for hiding Figure #
\usepackage{caption}

% Page formatting
\usepackage[letterpaper, margin=1in]{geometry}

% Package for nice syntax highlighting for code
\usepackage{minted}

% Allows line breaks in the math environment
\allowdisplaybreaks

\usepackage[activate={true,nocompatibility},final,tracking=true,kerning=true,spacing=true,factor=1100,stretch=10,shrink=10]{microtype}
\microtypecontext{spacing=nonfrench}
% activate={true,nocompatibility} - activate protrusion and expansion
% final - enable microtype; use "draft" to disable
% tracking=true, kerning=true, spacing=true - activate these techniques
% factor=1100 - add 10% to the protrusion amount (default is 1000)
% stretch=10, shrink=10 - reduce stretchability/shrinkability (default is 20/20)


\title{CS 310: Iterator}
\author{Connor Baker}
\date{February 7, 2019}

\begin{document}

\maketitle

\subsection*{List Implementation $\mathbf{O(n)}$ Summary}
\begin{center}
    \begin{tabular}{lcccccr} \toprule
        Implementation & get/set & add/del at end & add/del start & add/del mid & search & can grow? \\ \midrule
        Static Array & 1 & 1 & $N$ & $N$ & $N$ & no \\
        Dynamic Array & 1 & $\mathbf{1^*}$ & $N$ & $N$ & $N$ & no \\     Singly-Linked & $N$ & 1, $N$ & 1 & $N$ & $N$ & yes \\
        Doubly-Linked & $N$ & 1 & 1 & $N$ & $N$ & yes \\ \bottomrule
    \end{tabular}
    \begin{center}*Amortized analysis\end{center}
\end{center}


\subsection*{Review: Pytania Questions}
\begin{itemize}
    \item Dynamic arrays
    \item Linked lists
\end{itemize}


\subsection*{Problem: List Iteration}
\begin{minted}{java}
List<Intgers> l = ...;
int sum = 0;
for (int i = 0; i < size(); i++) {
    sum += l.get(i);
}
\end{minted}
\begin{itemize}
    \item What is the complexity of this loop?
    \begin{itemize}
        \item Array-based list
        \begin{itemize}
            \item $O(n)$
        \end{itemize}
        \item Linked list-based list
        \begin{itemize}
            \item $O(n^2)$
        \end{itemize}
    \end{itemize}
    \item Recall that
    \begin{itemize}
        \item \mintinline{java}{ArrayList.get(i)}: $O(1)$
        \item \mintinline{java}{LinkedList.get(i)}: $O(n)$
    \end{itemize}
\end{itemize}

\subsection*{\mintinline{java}{Iterator}s}
\begin{itemize}
    \item Data structures other than arrays have the following properties:
    \begin{itemize}
        \item Nontrivial get/set
        \item Must preserve some internal structure -- that means controlling access
        \item Element by element access takes time
    \end{itemize}
    \item Those qualities give rise to \mintinline{java}{Iterator}s
    \begin{itemize}
        \item A view of a data structure
        \item Supports efficient iterations on the collection
    \end{itemize}
\end{itemize}

\subsection*{Java \mintinline{java}{Iterators}}
\begin{itemize}
    \item \mintinline{java}{Interface Iterator<E>}
    \begin{itemize}
        \item Is in \mintinline{java}{java.util}
        \item http://docs.oracle.com/javase/8/docs/api/java/util/Iterator.html
    \end{itemize}
    \item Methods
    \begin{itemize}
        \item \textbf{Required:} \mintinline{java}{.hasNext()} and \mintinline{java}{.next()}
        \item Default: \mintinline{java}{.remove()} and \mintinline{java}{.forEachRemaining()}
    \end{itemize}
    \item Any class that implements the \mintinline{java}{List<E>} interface needs to implement these methods
    \begin{itemize}
        \item It follows then that \mintinline{java}{ArrayList}, \mintinline{java}{Vector}, and \mintinline{java}{Set} implement these
    \end{itemize}
\end{itemize}

\subsection*{Textbook Sample}
\begin{minted}{java}
package weiss.util;

/**
 * Iterator interface
 */
 public interface Iterator<AnyType> extends java.util.Iterator<AnyType> {
    /**
     * Tests if there are items not yet iterated over.
     */
    boolean hasNext();
    
    /**
     * Obtains the next (as yet unseen) item in the collection.
     */
    AnyType next();
    
    /**
     * Remove the last item returned by next.
     * Can only be called once after next.
     */
    void remove();
    }
 }
\end{minted}

\subsection*{Java List Iterators}
\begin{itemize}
    \item Sub-interface of \mintinline{java}{Iterator}: give access to a position in any \mintinline{java}{Collection}
    \item Different and more complex semantic: put a cursor \textbf{between} list items
    \item Operations
    \begin{itemize}
        \item Most important methods: \mintinline{java}{.next()} and \mintinline{java}{.hasNext()}
        \item Optional for general iterator: \mintinline{java}{.remove()}, \mintinline{java}{.previous()}, \mintinline{java}{.hasPrevious()}, and \mintinline{java}{.add()}
    \end{itemize}
\end{itemize}

\subsection*{Common Iterator Operations}
\begin{itemize}
    \item Use \mintinline{java}{.hasNext()} or \mintinline{java}{.hasPrevious()} to determine whether the end has been reached
    \item Use \mintinline{java}{.next()} or \mintinline{java}{.previous()} to move the position of the cursor (you can modify both to return the element that the cursor moved over)
    \end{itemize}
\begin{minted}{java}
List l = new List([A, B, C, D]) 
ListIterator itr = l.iterator() // [^A B C D]
itr.next() // [ A^B C D]
// A
itr.hasNext() // [ A^B C D]
// True
itr.next() // [ A B^C D]
// B
itr.previous() // [ A^B C D]
// B
itr.previous() // [^A B C D]
// A
itr.previous() // [^A B C D]
// NoSuchElementException
\end{minted}


\subsection*{\mintinline{java}{ListIterator} Operations}
\begin{itemize}
    \item \mintinline{java}{.add(x)} add \mintinline{java}{x} before whatever \mintinline{java}{.next()} would return
    \begin{itemize}
        \item Insert before the implicit cursor
    \end{itemize}
    \item \mintinline{java}{.remove()} removes the element which was returned from the last \mintinline{java}{.next()} or \mintinline{java}{.previous()} call
    \begin{itemize}
         \item It is \textbf{illegal} to remove an item without first calling \mintinline{java}{.next()} or \mintinline{java}{.previous()}
    \end{itemize}
\end{itemize}

\begin{minted}{java}
List l = new List([A, B, C, D]) 
ListIterator itr = l.iterator() // [^A B C D]
itr.add(X) // [X^A B C D] 
itr.next() // [X A^B C D]
// A
itr.remove() // [ X^B C D]
itr.remove() // [ X^B C D]
// IllegalStateException
itr.previous() // [^X B C D]
// X
itr.add(Y) // [Y^X B C D]
\end{minted}


\subsection*{Exercise}
\begin{itemize}
    \item Given a list and a sequence of operations, track the location of the iterator and the contents of the list
\end{itemize}

\subsection*{\mintinline{java}{Iterator} Implementation}
\begin{itemize}
    \item How do we code an iterator for a list?
    \begin{itemize}
        \item The implementation depends on the underlying data structure
        \item \mintinline{java}{ArrayList}, singly-linked list, and doubly-linked list should all provide their own unique iterator
    \end{itemize}
    \item \mintinline{java}{interface Iterator<E>}
    \begin{itemize}
        \item \mintinline{java}{java.util}
        \item \mintinline{java}{boolean hasNext()}
        \item \mintinline{java}{E next()}
        \item How does this compare with the \mintinline{java}{Iterator}s for \mintinline{java}{ArrayList}, \mintinline{java}{LinkedList} and so on?
    \end{itemize}
\end{itemize}

\subsection*{Java \mintinline{java}{Iterator}s}
\begin{itemize}
    \item \mintinline{java}{interface Iterable<T>}
    \begin{itemize}
        \item \mintinline{java}{java.lang}
        \item \mintinline{java}{Iterator<T> iterator()}
        \item https://docs.oracle.com/javase/8/docs/api/java/lang/Iterable.html
        \item The \mintinline{java}{List} is only one of its subinterfaces, and is implemented by \mintinline{java}{ArrayList}, \mintinline{java}{LinkdList}, and so on
    \end{itemize}
    \item What have we been missing so far?
    \begin{itemize}
        \item Well, when implementing our own \mintinline{java}{ArrayList} or \mintinline{java}{LinkedList}, we should have implemented \mintinline{java}{.iterator()}
        \item The return of \mintinline{java}{.iterator()} must be an \textbf{object} that can perform \mintinline{java}{.next()} and \mintinline{java}{.hasNext()} on our list
        \begin{itemize}
            \item But an object of which class?
            \begin{itemize}
                \item Anything that implements the iterator, so that it works for our class which will implement it
            \end{itemize}
            \item How and where should we define that class?
            \begin{itemize}
                \item We can use an anonymous class, or a static nested class (or, as a worse choice, an inner class).
            \end{itemize}
        \end{itemize}
    \end{itemize}
\end{itemize}

\subsection*{\mintinline{java}{Iterator} as a Separate Class}
\begin{minted}{java}
package weiss.ds;

public class MyContainer {
    Object[] items;
    int size;
    
    public Iterator iterator() {
        return new MyContainerIterator(this);
    }
    
    // Other methods not shown
}
\end{minted}

\begin{minted}{java}
// An interator class that steps through a MyContainer.

package weiss.ds;

class MyContainerIterator implements Iterator {
    private int current = 0;
    private MyContaner container;
    
    MyContainerIterator(MyContainer c) {
        container = c;
    }
    
    public boolean hasNext() {
        return current < container.size;
    }

    public Object next() {
        return container.items[current++];
    }
}
\end{minted}


\begin{itemize}
    \item Container class
    \begin{itemize}
        \item Issues?
        \begin{itemize}
            \item It's not encapsulated properly -- really, it should be a private class inside the container that it's iterating over.
        \end{itemize}
    \end{itemize}
\end{itemize}


\subsection*{\mintinline{java}{Iterator} as a Nested Class}
\begin{minted}{java}
package weiss.ds;

public class MyContainer {
    private Object[] items;
    private int size = 0;
    // Other methods for MyContainer not shown
    
    public Iterator iterator() {
        return new LocalIterator(this);
    }
    
    // The iterator class as a nested class
    private static class LocalIterator implements Iterator {
        private int current = 0;
        private MyContaner container;
        
        MyContainerIterator(MyContainer c) {
            container = c;
        }
        
        public boolean hasNext() {
            return current < container.size;
        }
    
        public Object next() {
            return container.items[current++];
        }
    }
}
\end{minted}

\subsection*{Why use a Nested Class?}
\begin{itemize}
    \item Main issue: we don't want to expose details we don't have to to the outside world
    \begin{itemize}
        \item The container class has to reveal details to \mintinline{java}{Iterator} (non-private data)
        \item The \mintinline{java}{Iterator} class has to be accessible to the container class (package visible)
    \end{itemize}
    \item Nested (private) iterator class: better encapsulation
    \begin{itemize}
        \item Considered as a member of the container
        \item Can access private data of the container
    \end{itemize}
\end{itemize}

\subsection*{\mintinline{java}{Iterator} as an Inner Class}
\begin{minted}{java}
package weiss.ds;

public class MyContainer {
    private Object[] items;
    private int size = 0;
    // Other methods for MyContainer not shown
    
    public Iterator iterator() {
        return new LocalIterator(this);
    }
    
    // The iterator class as an inner class
    private class LocalIterator implements Iterator {
        private int current = 0;
        
        public boolean hasNext() {
            return current < MyContainer.this.size;
        }
    
        public Object next() {
            return MyContainer.this.items[current++];
        }
    }
}
\end{minted}

\subsection*{Why use an Inner Class?}
\begin{itemize}
    \item Both nested and inner classes allow us to do the following:
    \begin{itemize}
        \item Put multiple classes in a single file
        \item Give access to the namespace of \mintinline{java}{OuterClass}
        \item Have access to the private methods of \mintinline{java}{OuterClass}
    \end{itemize}
    \item What's the difference between an inner class and a nested class?
    \begin{itemize}
        \item A nested class can be \mintinline{java}{static}
        \item Inner classes, upon construction, implicitly reference the instance of the outer class which caused its construction (\mintinline{java}{MyContainer.this})
        \begin{itemize}
            \item The constructor can also be dropped (the \mintinline{java}{MyContainer.this} reference is optional)
        \end{itemize}
    \end{itemize}
\end{itemize}
\begin{minted}{java}
// The iterator class as an inner class
private class LocalIterator implements Iterator {
    private int current = 0;
    
    public boolean hasNext() {
        return current < size;
    }
    
    public Object next() {
        return items[current++];
    }
}
\end{minted}




\subsection*{More Textbook Examples}
\begin{itemize}
    \item Singly linked node, list, and iterator (17.1-2)
    \begin{itemize}
        \item All classes are separate entities!
        \begin{itemize}
            \item \mintinline{java}{weiss.nonstandard.LinkedList}
            \item \mintinline{java}{weiss.nonstandard.ListNode}
            \item \mintinline{java}{weiss.nonstandard.LinkedListIterator}
        \end{itemize}
    \end{itemize}
    \item Doubly linked list (17.3-5)
    \begin{itemize}
        \item \mintinline{java}{weiss.util.LinkedList}
        \item \mintinline{java}{weiss.util.LinkedList.LinkedListIterator} (inner, non-\mintinline{java}{static})
        \item \mintinline{java}{weiss.util.LinkedList.Node} (nested, \mintinline{java}{static})
    \end{itemize}
\end{itemize}



\subsection*{Typical Anonymous Class Style}
\begin{minted}{java}
class MyList<T> implements Iterable<T> {
    public Iterator<T> iterator() {
        return new Iterator<T>() {
            public boolean hasNext() {
                // Implementation
            }
            public T next() {
                // Implementation
            }
        };
    }
    ...
}
\end{minted}

\subsection*{Two Ways to use an \mintinline{java}{Iterator}}
\begin{minted}{java}
public static void main main( String[] args ) {
    MyList<String> list = new MyList<>();
    list.add("Alpha");
    list.add("Bravo");
    list.add("Charlie");
    list.add("Delta");
    Iterator<String> iter = list.iterator();
    while(iter.hasNext()) {
        String item = iter.next();
        System.out.println(item);
    }
    for (String item : list) {
        System.out.println(item);
    }
}
\end{minted}


\subsection*{\mintinline{java}{ConcurrentModificationException}}
\begin{itemize}
    \item Our implementation doesn't try to coordinate multiple iterators changing the structure at the same time. \textbf{Note}: this is a different issue compared to access from different threads (we will not be covering that).
    \item Easy for reading and viewing
    \item Difficult for modification
\end{itemize}
\begin{minted}{java}
Iterator itr1 = list.iterator();
Iterator itr2 = list.iterator();
itr1.remove();
itr2.next(); // Error!
\end{minted}


\subsection*{Practice Problems}
\begin{itemize}
    \item Add \mintinline{java}{.iterator()} into our class examples of \mintinline{java}{ArrayList} and \mintinline{java}{LinkedList}
    \begin{itemize}
        \item Pay attention to the details -- it's always the edge cases that'll get ya
        \begin{itemize}
            \item Where does it point when it is created?
            \item What about \mintinline{java}{.add()} and \mintinline{java}{.remove()}?
        \end{itemize}
        \item What methods might be difficult in terms of time or algorithmic complexity for a singly linked list?
        \begin{itemize}
            \item Any algorithms which involve adding or deleting at the end of the linked list.
        \end{itemize}
    \end{itemize}
\end{itemize}


\subsection*{Summary}
\begin{itemize}
    \item \mintinline{java}{Iterator}s
    \begin{itemize}
        \item Support efficient traversing of all elements in the data structure
        \item Implementation details are different, but \mintinline{java}{Iterator} and \mintinline{java}{Iterable} provide a common \mintinline{java}{interface}
    \end{itemize}
    \item Next lecture: \mintinline{java}{Stack} and \mintinline{java}{Queue}
    \begin{itemize}
        \item Reading: Chapter 6, Chapter 16
    \end{itemize}
\end{itemize}




\end{document}